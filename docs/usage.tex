\documentclass{article}[12pt]
\fontfamily{georgia}

\usepackage{geometry}
\usepackage{graphicx}
\usepackage{amsmath}

\title{Basic features and usage of the task planning app}
\author{Albert ZHANG Tongjun}

\begin{document}
\maketitle
\tableofcontents
~\\~\\


\section{Use Cases}
The application can be used as either a guest or a registered user.
The \textbf{data created by a guest} will be processed but \textbf{NOT STORED} by the server and \textbf{WILL BE LOST} once the session is closed or expired.
However, a guest can choose to download their planned events onto their own device (in .ics format).
Users can also import their events using from files (.ics format only, only basic VEVENT supported).
~\\~\\
At the current stage, registration will not require verification of any mobile number or emails.
Since the registration process does not require contact verification, the password cannot be recovered once forgotten.
~\\~\\
The application will have the several following modes.
~\\

\subsection{Calendar Mode}
In this mode, the application will behave like a normal online calendar.
It will display one month per page.
~\\~\\
Users can create events and add them to their calendar.
Events can be set to ``parallelable'' or ``non-parallelable''.
When user tries to add a ``non-parallelable'' event that clashes with another event, a warning will be generated.
~\\

\subsection{Day-Planner Mode}
In this mode, the application will show the events in a specific day.
It will also list out any interval between two events.
~\\~\\
A section will be provided to list out all of the events on that day so that the user can use it as a to-do list.
~\\

\subsection{To-Do List Mode}
A list of all the events in the future will be provided for use as a to-do list.
~\\



\section{Backend}
The backend will be implemented using Go.
It will receive http requests send by the client and make response accordingly.
~\\~\\
Data generated by users will be stored using a mysql database.
~\\~\\
The application will be designed to be deployed on a Linux server (Ubuntu 20.04 LTS).
A cloud server with fixed IP will be used for deployment during testing.



\section{Others}
The network traffic will not be protected with TLS.
Thus, the validation credentials during login will be processed with hash values instead of orginal text.
~\\~\\
The frontend will be designed primarily for use on computers (landscape screens).
There will be a light mode and a dark mode, with light mode being the default option.
\end{document}

